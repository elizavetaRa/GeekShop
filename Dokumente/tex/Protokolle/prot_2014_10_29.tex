\documentclass{scrartcl}
 
\usepackage[utf8]{inputenc}
\usepackage[T1]{fontenc}
\usepackage{lmodern}
\usepackage[ngerman]{babel}
\usepackage{booktabs,paralist}
\usepackage{scrpage2,lastpage}

\rofoot{\thepage~}

\begin{document}
\begin{center}
\LARGE \bf{Protokoll vom 29.10.2014 \\
SWT14W30}
\end{center}

\begin{tabular}{rp{10cm}}
Protokollant & Marcus Kammerdiener \\
Satz & Felix Döring \\
Anwesend & Sebastian Döring, Felix Döring, Marcus Kammerdiener,\\
& Dominik Lauck, Elizaveta Ragozina \\
\end{tabular}

\vspace*{3em}

\section{Ablauf}
\subsection{Kundengespr\"ach}
\subsubsection{Passw\"orter}
Bezüglich der Passwörter wurde gesagt, dass auch die Angestellten selbst ihre Passwörter ändern können. Allerdings
bekommt der Ladenbesitzer dann eine Benachrichtigung über den Vorgang. Auch soll der Ladenbesitzer in der Lage sein, Passwörter von Angestellten zu ändern. Das ursprüngliche Passwort des Angestellten soll er jedoch nie zu Gesicht bekommen. Nach dem Änderungsvorgang soll das neue Passwort ausgegeben werden, damit der Ladenbesitzer es an seinen Angestellten weitergeben kann.\\
Der Ladenbesitzer soll beim Aufsetzen der Software u.\,a. die Passwortregeln festlegen können. Auch soll es bei einem
Update des Systems möglich sein, die Regeln zu ändern. Nach dem Update sollen alle Angestellten eine Benachichtigung erhalten, ihr Passwort zu ändern. Sollte ihr Passwort den neuen Sicherheitsbestimmungen nicht genügen, soll ein Login unmöglich sein, bis eine neues Passwort erstellt wurde, das den Regeln entspricht. Es wurde fesgelegt, dass der Ladenbesitzer gleichzeitig auch Admin des Systems ist.
\subsubsection{Reklamation}
Die Reklamation soll grundlegend so ablaufen: Der Kunde kommt in den Laden und meldet eine Reklamation an. Er gibt
dem Angestellten den Artikel und zeigt die dazugehörige Rechnung vor. Der Angestellte verifiziert dann, ob die
14-Tage-Frist eingehalten wurde. Sollte dem so sein, legt er den Artikel in den Reklamations-Warenkorb. Dieser 
wird dann vom Ladenbesitzer noch einmal geprüft. Sollte dieser grünes Licht geben, wird dem Kunden sein Geld 
erstattet. Der Ladenbesitzer bestimmt dann über die weitere Nutzung des Artikels.
\subsubsection{Neuer Angestellter}
Für einen neuen Angestellten legt der Ladenbesitzer ein neues Profil an. Für dieses Profil wird ein Passwort generiert und ausgegeben. Der Ladenbesitzer übergibt dann dem Angestellten das Passwort. Für einen vollständigen Login muss der neue Angestellte aber im Anmeldebildschirm ein eigenes Passwort erstellen, das den Passwortregeln entspricht.
\subsection{Konsultation}
In der Konsultation wurden die ersten Forschritte des Teams vorgestellt. Es wurde ein Prototyp für ein Anwendungsfalldiagramm
vorgestellt, an dem auch gleich Änderungen nach Kundenwünschen vorgenommen wurden. 
Für das Pflichtenheft wurden 
als Kann-Kriterium ein System für sich nicht wiederholende Witze beim Login genannt und als Muss-Kriterium die
Passwortverschlüsselung. 
Für den zu erstellenden kleinen Prototyp wurde gesagt, dass es sich um eine Web-Anwendung
handeln muss. Es ist ausreichend, ein einfaches Textfeld zu erstellen, das beim Speichern den Eintrag auf der
Seite anzeigt.
Bis zum Ende der Woche zu erstellen sind die Prototypen und die grundlegenden Diagramme. Es soll zu jedem Vorgang
der Software ein Sequenzdiagramm erstellt werden, es sei denn, es handelt sich nur um eine einzige ausgelöste Aktion.

\vspace*{1em}
\section{Aufgaben}
Sowohl das Pfichtenheft (mitsamt aller Diagramme) als auch die Prototypen sollen bis Ende der Woche fertiggestellt werden. Die Prototypen sollen bis Sonntag, 02.11.2014, 24 Uhr im Repository liegen, das Pflichtenheft bis zu diesem Zeitpunkt auf der Website zu finden sein.
\end{document}