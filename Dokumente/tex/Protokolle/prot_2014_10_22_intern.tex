\documentclass{scrartcl}
 
\usepackage[utf8]{inputenc}
\usepackage[T1]{fontenc}
\usepackage{lmodern}
\usepackage[ngerman]{babel}
\usepackage{booktabs,paralist}
\usepackage{scrpage2,lastpage}

\rofoot{\thepage~}

\begin{document}
\begin{center}
\LARGE \bf{Protokoll vom 22.10.2014 \\
SWT14W30 (intern)}
\end{center}

\begin{tabular}{rp{10cm}}
Protokollant & Marcus Kammerdiener \\
Satz & Felix Döring \\
Anwesend & Sebastian Döring, Felix Döring, Marcus Kammerdiener,\\
& Dominik Lauck, Elizaveta Ragozina \\
\end{tabular}

\vspace*{3em}

\section{Ablauf}
\subsection{Treffen}
Zunächst einigte sich die Gruppe darauf, zukünftig den Termin für das nächste Treffen am Ende des Protokolls zu vermerken. Es wurden erste Vorschläge für weitere Treffen unterbreitet, die Festlegung wurde jedoch aufgrund von Problemen mit dem Finden eines gemeinsamen Termins verschoben.
\subsection{Software}
Im Anschluss wurde die benötigte Software eingestellt. Zum einen wurden die Gruppenmitglieder für das GitHub-Repository freigeschaltet, zum anderen wurden alle nötigen Einstellungen an der genutzten IDE (IntelliJ IDEA) vorgenommen, um einen reibungslosen Ablauf zu garantieren. Die Entwicklungsumgebung wurde außerdem an GitHub gekoppelt. Aufgrund von Schwierigkeiten bei der Einstellung wurde die Fertigstellung auf später verschoben.
\subsection{Shop}
Die Gruppe fing nun an, Ideen für die Software auszutauschen. Zuerst wurde nach einem Namen gesucht, allerdings wurde festgelegt, vorerst bei "`Think Nerd"' zu bleiben. Danach wurden erste Ideen zu benötigten Diagrammen zusammengetragen. Es entstand der Prototyp eines Kontextdiagramms, die Erstellung eines Klassendiagramms wurde auf später verschoben. Im Folgenden kamen Fragen zum generellen Verständnis auf, die in der anschlie\ss{}enden Konsultation geklärt werden sollten, so zum Beispiel die Frage nach den Rechten der Angestellten und des Ladenbesitzers.
\section{N\"achstes Treffen}
Der nächste Termin für ein Treffen wurde auf Mittwoch, den 29.10.2014, zur 3.~DS festgelegt.
\end{document}