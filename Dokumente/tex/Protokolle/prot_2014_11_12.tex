\documentclass{scrartcl}
 
\usepackage[utf8]{inputenc}
\usepackage[T1]{fontenc}
\usepackage{lmodern}
\usepackage[ngerman]{babel}
\usepackage{booktabs,paralist}
\usepackage{scrpage2,lastpage}

\rofoot{\thepage~}

\begin{document}
\begin{center}
\LARGE \bf{Protokoll vom 05.11.2014 \\
SWT14W30}
\end{center}

\begin{tabular}{rp{10cm}}
Protokollant & Marcus Kammerdiener \\
Satz & Felix Döring \\
Anwesend & Sebastian Döring, Felix Döring, Marcus Kammerdiener,\\
& Dominik Lauck, Elizaveta Ragozina \\
\end{tabular}

\vspace*{3em}

\section{Ablauf}
\subsection{Vorstellung des Entwurfsklassendiagramms}
Zu Beginn der Konsultation wurde das Entwurfsklassendiagramm der Gruppe vorgestellt. Der Kunde war mit der Ausführung zufrieden, hatte allerdings einige Anmerkungen, so wurde zum Beispiel festgestellt, dass die Bezahlmöglichkeit Scheck unnötig ist und weggelassen werden kann. Die Gruppe fragte, ob alle Methoden in das Diagramm gehören, allerdings wären nur die nötig, die höchstwahrscheinlich in das Programm integriert werden sollen.
\subsection{Kategorien}
Das nächste Große Thema war die Frage zur Umsetzung der Produktkategorien. Die ursprüngliche Idee, dass es beliebig viele Ebenen von Kategorien geben kann, wurde als extrem aufwendig eingestuft und der Kunde zu Vereinfachungsmöglichkeiten befragt. Am Ende wurde beschlossen, dass eine Ober- und Unterkategorie zu einen Produkt ausreichend ist.
\vspace*{1em}

\section{Aufgaben}
um nächsten Termin sind fertigzustellen: die bearbeiteten Sequenzdiagramme, die Zustandsmaschine und der GUI Prototyp.
der nächste Termin ist Dienstag, der 18.11.2014, zur 6.DS.

\end{document}