\documentclass{scrartcl}
\usepackage{geometry}
\geometry{a4paper, top=30mm, left=30mm, right=30mm, bottom=30mm, headsep=15mm, footskip=15mm}
 
\usepackage[utf8]{inputenc}
\usepackage[T1]{fontenc}
\usepackage{lmodern}
\usepackage{tabularx}
\usepackage{multirow}
\usepackage{paralist}
\usepackage{enumitem}
\usepackage[ngerman]{babel}
\usepackage{booktabs,paralist}
\usepackage{scrpage2,lastpage}

\parindent 0pt

\makeatletter
\newcommand{\minipagetrue}{\@minipagetrue}
\setlist[itemize]{itemsep=0pt,after=\vskip-\baselineskip,label={\textbullet},leftmargin=*,before=\minipagetrue,}

\rofoot{\thepage~}
\begin{document}
\begin{center}
\LARGE \bf{Protokoll vom 21.01.2015 \\
SWT14W30}
\end{center}

\begin{tabular}{rp{10cm}}
Protokollant & Marcus Kammerdiener \\
Satz & Sebastian Döring \\
Anwesend & Sebastian Döring, Felix Döring, Marcus Kammerdiener,\\
& Dominik Lauck, Elizaveta Ragozina \\
\end{tabular}

\vspace*{3em}

\section{Ablauf}
Zu Beginn der Konsultation wurde der Online-Fragebogen zur Evaluation des Praktikums gemeinsam mit dem Tutor ausgefüllt.

Anschließend wurden noch Details zur Abschlusspräsentation abgesprochen.
Die eigentliche Präsentation soll 20 Minuten betragen. Sie untergliedert sich wie folgt:
\begin{itemize}
\item 2--3 Minuten: Vorstellung der Gruppe und der Aufgabe
\item 7--8 Minuten: Vorstellung des Entwicklungsprozesses: Wie lag die Gruppe in der Zeit? Wo gab es Probleme? Was würde die Gruppe nächstes Mal anders machen?
\item 10 Minuten: Produktvorführung: Vorstellung der Muss-Kriterien, Präsentation von Kann-Kriterien
\end{itemize}
Anschließend findet eine zehnminütige Diskussion statt, in der Fragen jedweder Art an die Gruppe gestellt werden können.

Zudem meinte der Tutor auf Nachfrage, dass Diagramme zur Präsentation nicht nötig sind. Sie können jedoch zur Veranschaulichung des Entwicklungsprozesses genutzt werden.
\end{document}