\documentclass{scrartcl}
 
\usepackage[utf8]{inputenc}
\usepackage[T1]{fontenc}
\usepackage{lmodern}
\usepackage[ngerman]{babel}
\usepackage{booktabs,paralist}
\usepackage{scrpage2,lastpage}

\rofoot{\thepage~}

\begin{document}
\begin{center}
\LARGE \bf{Protokoll vom 18.11.2014 \\
SWT14W30}
\end{center}

\begin{tabular}{rp{10cm}}
Protokollant & Marcus Kammerdiener \\
Satz & Felix Döring \\
Anwesend & Sebastian Döring, Felix Döring, Marcus Kammerdiener,\\
& Dominik Lauck, Elizaveta Ragozina \\
\end{tabular}

\vspace*{3em}

\section{Ablauf}
\subsection{Diagramme}
Das heutige Treffen begann mit der Übermittlung einer Nachricht von Frau Demuth: Microsoft Visio ist als UML-Tool nicht zulässig. Es fehlen grundlegende Eigenschaften eines UML-Tools. So können mit Visio beispielsweise Verbindungslinien ins Leere gezogen werden, was ein UML-Tool unterbindet. Start- und Endpunkt einer Verbindunslinie muss immer eine UML-Komponente sein. Ferner ist es in Visio nicht möglich, mehrere Diagramme in einem Projekt zusammenzufassen. Das Entwurfsklassendiagramm muss daher noch einmal mit MagicDraw gezeichnet werden. \\
Es wurde auch darauf hingewiesen, dass bei den Sequenzdiagrammen auf Konsistenz mit dem Entwurfsklassendiagramm zu achten ist.
\subsection{Protokolle}
Betreffs der Protokolle wurde angewiesen, zu notieren, welches Gruppenmitglied was gemacht hat und ob es Fragen oder Probleme gab.
\subsection{Entwurfsklassendiagramm}
Das Entwurfsklassendiagramm soll als Vorlage zum Programmieren dienen. Die Dokumentationen von Spring und Salespoint sollen genutzt werden, um das Diagramm an die Frameworks anzupassen.
\subsection{Zwischenpräsentation}
Es wurden auch Informationen für die Zwischenpräsentation weitergegeben: Diese soll am 27.11.2014 in der 6.~DS stattfinden. Die Vorstellung des Prototyps ist Teil der Präsentation. Der Prototyp soll nur ein Klick-Dummy sein, um das grundlegende Aussehen des Projekts vorzustellen. \\
Einige Features sollen jedoch schon enthalten sein. So wären zum Beispiel ein Witze-Repository und ein Login denkbar. Außerdem sollen ausgewählte Diagramme vorgestellt sowie wesentliche Designentscheidungen verteidigt werden.
\subsection{Hinweise zur Implementation}
Jedes Teammitglied ist für das ausgiebige Testen seiner Klassen verantwortlich, d.~h. jeder schreibt die Tests für seine Klassen selbst. Der Tester hat jedoch zusätzlich die Aufgabe, Tests für die Beziehungen zwischen den Klassen zu schreiben, um die Funktionstüchtigkeit des Projekts im Ganzen zu gewährleisten. \\
Der Chefprogrammierer ist zudem dafür zuständig, dass ein guter und einheitlicher Programmierstil gepflegt wird. Außerdem steht er seinen Teammitgliedern bei Fragen und Problemen bezüglich der Implementation zur Seite. \\
Deshalb ist es erforderlich, Tester und Ladenbesitzer dementsprechend mit weniger Implementationsarbeit zu betrauen.
\vspace*{1em}

\section{Aufgaben}
Bis zur Zwischenpr\"asentation ist der Prototyp fertigzustellen.

\section{N\"achste Konsultation}
Aufgrund der Zwischenpräsentation entfällt die nächste Konsultation am 26.11.2014, sofern in der Zwischenzeit keine neuen Fragen oder Probleme seitens der Gruppe auftauchen.

\end{document}