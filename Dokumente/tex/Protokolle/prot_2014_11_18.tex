\documentclass{scrartcl}
 
\usepackage[utf8]{inputenc}
\usepackage[T1]{fontenc}
\usepackage{lmodern}
\usepackage[ngerman]{babel}
\usepackage{booktabs,paralist}
\usepackage{scrpage2,lastpage}

\rofoot{\thepage~}

\begin{document}
\begin{center}
\LARGE \bf{Protokoll vom 18.11.2014 \\
SWT14W30}
\end{center}

\begin{tabular}{rp{10cm}}
Protokollant & Marcus Kammerdiener \\
Satz & Felix Döring \\
Anwesend & Sebastian Döring, Felix Döring, Marcus Kammerdiener,\\
& Dominik Lauck, Elizaveta Ragozina \\
\end{tabular}

\vspace*{3em}

\section{Ablauf}
\subsection{Diagramme}
Das heutige Treffen begann mit einer Nachricht von Frau Demuth: Visio ist als UML-Tool nicht zulässig. Es besitzt grundlegende Eigenschaften eines UML-Tools nicht, so dürfen zum Beispiel Verbindungen nicht ins nirgendwo gezogen werden, was mit Visio jedoch möglich ist. Das Entwurfsklassendiagramm muss deshalb noch einmal mit Magic-Draw neu gezeichnet werden. Es wurde darauf hingewiesen, bei den Sequenzdiagrammen auf konsistenz mit dem Entwurfsklassendiagramm zu achten.
\subsection{Protokolle}
Für die Protokolle wurde darauf hingewiesen, dass notiert wird, wer was gemacht hat und ob es Fragen und Probleme gab.\subsubsection{Entwurfsklassendiagramm}
Das Entwurfsklassendiagramm soll als Vorlage zum Programmieren dienen. Die Dokumentationen von Spring und Salespoint sollen genutzt werden, um das Diagramm an die Frameworks anzupassen.
\subsubsection{Zwischenpräsentation}
Es wurden auch Informationen für die Zwischenpräsentation weitergegeben: Sie soll am 27.11.2014 in der 6.DS stattfinden. Die Vorstellung des GUI Prototyps ist Teil der Präsentation. Dieser Prototyp soll nur ein Klick-Dummy sein, um das grundlegende Aussehen des Projekts vorzustellen. Diese Oberfläche und ist nicht bindend und kann später noch geändert, aber auch übernommen werden. Einige Features sollen jedoch schon enthalten sein, so wäre zum Beispiel ein Witze- Repository oder ein Login möglich und gern gesehen. Außerdem sollen ausgewählte Diagramme vorgestellt, sowie bestimmte Designentscheidungen verteidigt werden.
\subsubsection{Weitere Entwicklung}
Für die spätere Entwicklung wurde darauf hingewiesen, dass egal wir die Arbeit verteilt wird, jeder seine eingenen Testfälle schreibt. Der Tester hat jedoch noch die Aufgabe, Tests für die Beziehungen zwischen den Klassen zu schreiben, um  die Funktionstüchtigkeit des Projekts selbst zu gewährleisten.
\vspace*{1em}

\section{Aufgaben}
Bis zur Zwischenpr\"asentation ist der Prototyp fertig zu stellen.

\end{document}