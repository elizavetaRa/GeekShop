\documentclass{scrartcl}
 
\usepackage[utf8]{inputenc}
\usepackage[T1]{fontenc}
\usepackage{lmodern}
\usepackage[ngerman]{babel}
\usepackage{booktabs,paralist}
\usepackage{scrpage2,lastpage}

\rofoot{\thepage~}

\begin{document}
\begin{center}
\LARGE \bf{Protokoll vom 05.11.2014 \\
SWT14W30}
\end{center}

\begin{tabular}{rp{10cm}}
Protokollant & Marcus Kammerdiener \\
Satz & Felix Döring \\
Anwesend & Sebastian Döring, Felix Döring, Marcus Kammerdiener,\\
& Dominik Lauck, Elizaveta Ragozina \\
\end{tabular}

\vspace*{3em}

\section{Ablauf}
\subsection{Pr\"asentation des Pflichtenheftes/Abschluss der Analysephase}
Zu Beginn wurde dem Kunden das Pflichtenheft präsentiert. Dieser machte einige Verbesserungsvorschläge: Die Entscheidung des Teams, den Ladenbesitzer und den Angestellten von einer abstrakten Klasse erben zu lassen, wurde als unn\"otig kompliziert eingeschätzt. Es ist ausreichend, die Klasse Ladenbesitzer von der Klasse Angestellter erben zu lassen. Des Weiteren wurde darauf hingewiesen, dass in den Sequenzdiagrammen die Stakeholder mit "`M\"annchen"' gekennzeichnet werden sollten. Die Funktion wurde jedoch in dem benutzten UML-Programm MagicDraw nicht gefunden und muss nachgebessert werden. Außerdem sollte das Anzeigen des Witzes in das entsprechende Sequenzdiagramm eingezeichnet werden. Das Sequenzdiagramm zur Anmeldung ist \"uberflüssig und kann entfernt werden. Außerdem soll die Anzahl der Sequenzdiagramme reduziert werden.\\
Zudem wurde vorgeschlagen, die Komponenten Produktauswahl, Warenkorb, Zahlung, Reklamation, Sortiment und Verkaufsstatistik der Top-Level-Architektur zu einer Komponente Produktverwaltung zusammenzufassen.
\subsection{Entwurfsphase}
Nach der Kontrolle des Pflichtenheftes wurden die weiteren Schritte besprochen.
\subsubsection{Entwurfsklassendiagramm}
Das Entwurfsklassendiagramm soll als Vorlage zum Programmieren dienen. Die Dokumentationen von Spring und Salespoint sollen genutzt werden, um das Diagramm an die Frameworks anzupassen.
\subsubsection{Weitere Diagramme}
Darüberhinaus sind die Sequenzdiagramme aus der Analysephase zu bearbeiten und an das Entwurfsklassendiagramm anzupassen. Außerdem soll eine Zustandsmaschine erstellt werden.
\subsubsection{Entwicklerdokumentation}
In der Entwurfsphase soll die Erstellung der Entwicklerdokumentation begonnen werden. Dies kann auch sp\"ater in der Phase geschehen, es sind Templates für die Erstellung verf\"ugbar.
\subsubsection{Gro\ss{}er Prototyp}
Der große Prototyp ist zum Ende der Entwurfsphase fertigzustellen. Es soll sich haupts\"achlich um einen Klick-Dummy handeln, der das Navigieren in der Anwendung simuliert. Es sollen nur grundlegende Funktionen implementiert werden, so z.\,B. der Login und das Anzeigen eines Witzes. Dafür sollen entsprechende Dummy-Daten schon im System enthalten sein.
\subsubsection{Zwischenpr\"asentation}
Die Zwischenpr\"asentation findet am Ende der Entwurfsphase statt. Dabei werden die Anforderungen, Analyse- und Entwurfsdiagramme und der große Prototyp vorgestellt. Die Pr\"asentation soll etwa 15 Minuten lang sein, mit einer anschließenden Fragerunde. Sie kann entweder mit Powerpoint oder Latex durchgeführt werden.
\vspace*{1em}

\section{Aufgaben}
Bis zur n\"achsten Konsultation fertigzustellen sind das Entwurfsklassendiagramm und die angepassten Sequenzdiagramme sowie die Zustandsmaschine.\\\\
Des Weiteren sollen alle Dokumente auch ins Repository geladen werden.

\end{document}