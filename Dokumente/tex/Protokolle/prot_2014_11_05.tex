\documentclass{scrartcl}
 
\usepackage[utf8]{inputenc}
\usepackage[T1]{fontenc}
\usepackage{lmodern}
\usepackage[ngerman]{babel}
\usepackage{booktabs,paralist}
\usepackage{scrpage2,lastpage}

\rofoot{\thepage~}

\begin{document}
\begin{center}
\LARGE \bf{Protokoll vom 05.11.2014 \\
SWT14W30}
\end{center}

\begin{tabular}{rp{10cm}}
Protokollant & Marcus Kammerdiener \\
Satz & Felix Döring \\
Anwesend & Sebastian Döring, Felix Döring, Marcus Kammerdiener, Dominik Lauck, Elizaveta Ragozina \\
\end{tabular}

\vspace*{3em}

\section{Ablauf}
\subsection{Kundengespr\"ach}
\subsubsection{Pflichtenheft}
Zu beginn wurde dem Kunden das Pflichtenheft vorgelegt. Dieser hatte einige Verbesserungsvorschläge: Die Entscheidung des Teams, den Ladenbesitzer und den Angestellten von einer abstrakten Klasse erben zu lassen wurde als unn\"otig kompliziert eingeschätzt. Es ist ausreichend die Klasse Ladenbesitzer von der Klasse Angestellter erben zu lassen. Des weiteren wurde darauf hingewiesen, dass in den Sequeznzdiagrammen die Stakeholder mit "`M\"annchen"' gekennzeichnet werden sollten. Die Funktion wurde jedoch in dem benutzten UML Programm nicht gefunden und muss nachgebessert werden. Außerdem sollte das Anzeigen des Witzes in das entsprechende Sequenzdiagramm eingezeichnet werden. Das Sequenzdiagramm zur Anmeldung ist \"uberflüssig und kann entfernt werden. Außerdem soll die Anzahl der Sequenzdiagramme reduziert werden.\\
Daf\"ur wurde vorgeschlagen, die Diagramme für Produktauswahl, Warenkorb, Zahlung, Reklamation, Sortiment und Verkaufsstatistik zu einem Diagramm unter dem Namen Produktverwaltung kombiniert werden.
\subsection{Konsultation}
Nach der Kontrolle des Pflichtenheftes wurde die Gruppe auf die weiteren Schritte vorbereitet.
\subsubsection{Entwurfsklassendiagramm}
Das Entwurfsklassendiagramm soll als Vorlage zum Programmieren dienen. Die Dokumentationen von Spring und Salespoint sollen genutzt werden, um das Diagramm anzupassen.
\subsubsection{Entwurfsphase}
In der Entwurfsphase soll die Erstellung der Entwicklerdokumentation erfolgen. Dies kann sp\"ater in der Phase geschehen, es sind Templates für die Erstellung verf\"ugbar. Außerdem soll eine Zustandsmaschine erstellt werden.
\subsubsection{Gro\ss{}er Prototyp}
Der große Prototyp ist zum Ende der Entwurfsphase fertigzustellen. Es soll sich haupts\"achlich um einen Klick - Dummy handeln, der das navigieren in der Anwendung simuliert. Es sollen nur grundlegende Funktionen implementiert werden, so zum Beispiel der Login und das anzeigen eines Witzes. Dafür sollen entsprechende Dummy - Daten schon im System enthalten sein.
\subsubsection{Zwischenpr\"asentation}
Die Zwischenpr\"asentation findet am Ende der Entwurfsphase statt. Dabei werden die Anforderungen, Analyse- und Entwurfsdiagramme und der große Prototyp vorgestellt. Die Pr\"asentation soll etwa 15 Minuten lang sein, mit einer anschließenden Fragerunde. Sie kann entweder mit Powerpoint oder Latex durchgeführt werden. Außerdem sind die Sequenzdiagramme zu bearbeiten und an das Entwurfsklassendiagramm anzupassen.
\vspace*{1em}

\section{Aufgaben}
Bis zur n\"achsten Konsultation fertigzustellen sind das Entwurfsklassendiagramm und die angepassten Sequenzdiagramme sowie die Zustandsmaschine.
\end{document}