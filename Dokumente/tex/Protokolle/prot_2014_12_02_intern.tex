\documentclass{scrartcl}
 
\usepackage[utf8]{inputenc}
\usepackage[T1]{fontenc}
\usepackage{lmodern}
\usepackage[ngerman]{babel}
\usepackage{booktabs,paralist}
\usepackage{scrpage2,lastpage}

\rofoot{\thepage~}

\begin{document}
\begin{center}
\LARGE \bf{Protokoll vom 02.12.2014 \\
SWT14W30 (intern)}
\end{center}

\begin{tabular}{rp{10cm}}
Protokollant & Marcus Kammerdiener \\
Satz & Felix Döring \\
Anwesend & Sebastian Döring, Marcus Kammerdiener, Dominik Lauck,\\
& Elizaveta Ragozina \\
\end{tabular}

\vspace*{3em}

\section{Ablauf}
Das heutige Treffen diente hauptsächlich der Beantwortung einiger Fragen betreffs Implementation und fand aus diesem Grund im Lernraum statt. \\
Zum einen wurde die Frage gestellt, ob Passwörter in SalesPoint verschlüsselt in der Datenbank gespeichert werden und, falls dies nicht der Fall sein sollte, wie die Verschlüsselung umzusetzen wäre. Hintergrund der Frage ist der Wunsch des Kunden, Passwörter verschlüsselt zu speichern. Als Antwort wurde gegeben, dass die Verschlüsselung in SalesPoint schon realisiert ist. \\
Außerdem wurde gefragt, wie die Laufzeitdatenbank der Anwendung in eine persistente Datenbank überführt werden kann. Die Umsetzung wurde anhand des Videoshops gezeigt und in das eigene Projekt integriert.

\vspace*{1em}

\section{N\"achstes Treffen}
Das nächste Gruppentreffen findet wie gewohnt am Mittwoch, dem 03.12.2014, zur 3.\,DS statt.
\end{document}