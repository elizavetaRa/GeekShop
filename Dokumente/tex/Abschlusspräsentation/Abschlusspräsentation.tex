\documentclass[]{tudbeamer}
\usepackage{graphicx}
\usepackage{listings}
\usepackage{menukeys}
\usepackage{ngerman}
\usefonttheme{}

\begin{document}
\title[Abschlusspr\"asentation Geek-Shop]{Abschlusspr\"asentation Geek-Shop}
\subtitle{Softwaretechnologie-Projekt 2014 Gruppe 30}
\date{22.01.2015}

\maketitle

\begin{frame}
\tableofcontents
\end{frame}

\section{Gruppenmitglieder}
\begin{frame}
\frametitle{Gruppenmitglieder}
\begin{tabular}{rp{10cm}}
Chefprogrammierer: & Sebastian D\"oring \\
Assistent: & Elizaveta Ragozina \\
Sekret\"ar: & Marcus Kammerdiener \\
Tester: & Dominik Lauck \\
Administrator: & Felix D\"oring \\
\end{tabular}
\end{frame}

\section{Aufgabenstellung}
\begin{frame}
\frametitle{Aufgabenstellung}
\begin{itemize}
\item An- und Abmeldesystem
\item Passwortverwaltung
\item Benutzerverwaltung
\item Witzverwaltung
\item Sortiment- und Lagerverwaltung
\item Warenkorbverwaltung
\item Verkaufsabwicklung
\item XML-Exportfunktion f\"ur Verkaufsdaten
\end{itemize}
\end{frame}

\section{Entwicklungsprozess}
\begin{frame}
\frametitle{Entwicklungsprozess}
\begin{tabular}{p{.3\textwidth}|p{.3\textwidth}|p{.3\textwidth}}
Projektphase & Zeitraum & Dauer\\ \hline
Einarbeitung & 14.10. - 19.10. & 6 Tage\\
Analyse & 20.10. - 02.11. & 14 Tage\\
Entwurf & 03.11. - 23.11. & 21 Tage\\
Implementation \& Test & 24.11. - 14.12. & 21 Tage\\
Wartung \& Pflege & 15.12. - 18.01. & 21 Tage\\
\end{tabular}
\end{frame}

\section{\"Anderungen}
\begin{frame}
\frametitle{\"Anderungen}
\begin{itemize}
\item Composite-Pattern sollte f\"ur die Kategorien verwendet werden
\begin{itemize}
\item einzelne Klassen implementiert, da sie verschiedene Eigenschaften besitzen
\end{itemize}
\item Order sollte einfach als Reklamation deklariert werden
\begin{itemize}
\item wegen SalesPoint nicht möglich
\item eigene Klasse reclaimOrder implementiert, die mit der ursprünglichen Order verbunden wird
\end{itemize}
\end{itemize}
\end{frame}

\end{document}
