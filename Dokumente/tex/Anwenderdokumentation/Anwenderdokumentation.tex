\documentclass[12pt,a4paper]{article}
\usepackage{geometry}
\geometry{a4paper, top=19mm, left=30mm, right=30mm, bottom=20mm,headsep=10mm, footskip=10mm}
\usepackage{fontspec}
\usepackage{lmodern}
\usepackage{color}
\usepackage{tabularx}
\usepackage{graphicx}
\usepackage{slantsc}
\usepackage[ngerman]{babel}
\usepackage{booktabs,paralist}
\usepackage{scrpage2,lastpage}
\usepackage{menukeys}
\usepackage{tocloft}
% Unserer Blau-Farbe definieren
\definecolor{geekblue}{rgb}{0.01, 0.67, 0.99}
\definecolor{geekblue2}{rgb}{0.01, 0.40, 0.99}
% Farbe des Menus aendern
\changemenucolor{gray}{bg}{named}{geekblue}
\changemenucolor{gray}{br}{named}{geekblue}
\changemenucolor{gray}{txt}{named}{white}
\usepackage[colorlinks, breaklinks, linkcolor=geekblue2, menucolor=black, pagecolor=geekblue2, urlcolor=geekblue2]{hyperref}
% Hurenkinder und Schusterjungen verhindern
\clubpenalty=10000
\widowpenalty=10000
\displaywidowpenalty=10000
% Keine Einrückung bei Absatzen
\parindent 0pt
%
%
%
%
%
%        TYPESETTING ENGINGE MUST BE XeLaTeX!
%
%
%
%
%
% Define new Variables for section
\newcommand{\getsection}{undefined}
\newcommand{\setsection}[1]{\renewcommand{\getsection}{#1}}
\newcommand{\getnumber}{0}
\newcommand{\setnumber}[1]{\renewcommand{\getnumber}{#1}}
% Define new Variables for subsection
\newcommand{\getsubsection}{undefined}
\newcommand{\setsubsection}[1]{\renewcommand{\getsubsection}{#1}}
\newcommand{\getsubnumber}{0}
\newcommand{\setsubnumber}[1]{\renewcommand{\getsubnumber}{#1}}
% Define new Variables for subsubsection
\newcommand{\getsubsubsection}{undefined}
\newcommand{\setsubsubsection}[1]{\renewcommand{\getssubsubection}{#1}}
\newcommand{\getsubsubnumber}{0}
\newcommand{\setsubsubnumber}[1]{\renewcommand{\getsubsubnumber}{#1}}
%
%
% Define \menusection
%
% Usage: \menusection{Number of Section}{Name of Section}
\def\menusection#1#2{\newpage\addcontentsline{toc}{section}{\protect\numberline{#1} #2}\setnumber{#1}\setsection{#2}{\fontsize{30}{30}\menu{\hspace*{.2cm}#2\hspace*{.2cm}}}\vspace{.5cm}\par}
%
%
% Define \menusubsection
%
% Usage: \menusubsection{Number of Subsection}{Name of Subsection}
\def\menusubsection#1#2{\vspace{.7cm}\addcontentsline{toc}{subsection}{\protect\numberline{\getnumber.#1} #2}\setsubnumber{#1}\setsubsection{#2}{\LARGE{\menu{\hspace*{.15cm}\getsection\hspace*{.15cm} > \hspace*{.15cm}#2\hspace*{.15cm}}}}\par\vspace{.5cm}}
%
%
% Define \menusubsubsection
%
% Usage: \menusubsubsection{Number of Subsubsection}{Name of Subsubsection}
\def\menusubsubsection#1#2{\vspace{.7cm}\addcontentsline{toc}{subsubsection}{\protect\numberline{\getnumber.\getsubnumber.#1} #2}{\Large{\menu{\hspace*{.1cm}\getsection\hspace*{.1cm} > \hspace*{.1cm}\getsubsection\hspace*{.1cm} > \hspace*{.1cm}#2\hspace*{.1cm}}}}\par\vspace{.5cm}}



\begin{document}
%
% Titelblatt
\begin{titlepage}
\begin{center}
\textsc{\LARGE Technische Universit\"at Dresden} \\[0.5cm]
\textsc{\LARGE Softwareprojekt Gruppe 30}\\[0.5cm]
\textsc{\LARGE Geek-Shop}\\[3cm]
{\Huge \bfseries Pflichtenheft}\\
\vspace*{\fill}
Sebastian D\"oring (Chefprogrammierer), Felix D\"oring (Administrator), Marcus Kammerdiener (Sekret\"ar), Dominik Lauck (Testverantwortlicher), Elizaveta Ragozina (Assistent)\\[0.5cm]
\url{http://is63050.inf.tu-dresden.de/~swt14w30/index.php}
\end{center}
\end{titlepage}

%
% Inhaltsverzeichnis
\begingroup
\hypersetup{linkcolor=black}
\cftsetindents{section}{0em}{1.15em}
\cftsetindents{subsection}{1.5em}{2em}
\cftsetindents{subsubsection}{3.85em}{2.7em}
\tableofcontents
\endgroup
\newpage


% % % Anmeldung
\phantomsection\menusection{1}{Anmeldung}
\label{Anmeldung}
Da die Anwendung für die Kasse konzipiert ist, kann die Software nur nach erfolgreicher Anmeldung verwendet werden. Damit wird die Nutzung durch Unbefugte unterbunden.


% % Authentifikation
\phantomsection\menusubsection{1}{Authentifikation}
\label{Authentifikation}
Die Anmeldung erfolgt durch Eingabe von Benutzername und Passwort.
Für den ersten Login ist dem Ladenbesitzers (Benutzername \texttt{owner}) das Passwort \texttt{123} zugewiesen. Dieses sollte nach erstmaliger Anmeldung den Sicherheitsregeln für Passwörter entsprechend \hyperref[Passwort ändern]{geändert} werden.

Unter bestimmten Umständen bekommt ein Angestellter nach der Anmeldung eine Meldung angezeigt, in der er aufgefordert wird, sein Passwort den Sicherheitsregeln anzupassen. Dies kann zwei Ursachen haben:
\begin{enumerate}
\itemsep 0pt
\item Der Angestellte hat sich zum ersten Mal mithilfe des Passworts, das der Ladenbesitzer dem Angestellten übermittelt hat, angemeldet.
\item Die \hyperref[Passwortregeln ändern]{Sicherheitsregeln für Passwörter wurden geändert} und das alte Passwort des Angestellten entspricht nicht mehr den neuen Vorschriften.
\end{enumerate}

In beiden Fällen muss der Angestellte sich ein neues Passwort zulegen, das den Passwortregeln genügt.
Um sicherzugehen, muss das neue Passwort nochmals eingegeben werden. Beide Passwörter müssen übereinstimmen.

Um die Passwortänderung auszuführen, fehlt lediglich der Klick auf den Button \textit{Passwort ändern}. Wenn keine Fehlermeldungen an den Eingabefeldern erscheinen, wurde das Passwort erfolgreich geändert.

% % Begrüßungswitz
\phantomsection\menusubsection{2}{Begrüßungswitz}
\label{Begrüßungswitz}
Nach erfolgreicher Anmeldung bekommt der Nutzer einen Nerdwitz zur Begrüßung angezeigt.

In der Navigationsleiste sind bereits die Schaltflächen der verschiedenen Kassenfunktionen übersichtlich aufgelistet.


% % Abmeldung
\phantomsection\menusubsection{3}{Abmeldung}
\label{Abmeldung}
Um sich wieder abzumelden, genügt der Klick auf die Schaltfläche \textit{Logout} an letzter Stelle in der Navigationsleiste.
\newpage


% % % Artikelsuche
\phantomsection\menusection{2}{Artikelsuche}
\label{Artikelsuche}
Eine zentrale Funktion der Kasse ist die Artikelsuche, welche eine Liste von allen gefundenen Artikeln, die im \hyperref[Sortimentverwaltung]{Sortiment} vorhanden sind, liefert.


% % Suchen
\phantomsection\menusubsection{1}{Suchen}
\label{Suchen}
Es können Artikel nach Bezeichnung und Artikelnummer sowie Kategorien gesucht werden. Geben Sie dazu einfach den betreffenden Suchbegriff in das Suchfeld ein und klicken Sie auf \textit{Suchen} oder betätigen Sie die Eingabetaste.

Falls die Suche keine Resultate liefert, wird eine entsprechende Meldung ausgegeben.

Wenn nichts eingegeben und auf Suchen geklickt wurde, werden alle im Sortiment befindlichen Artikel angezeigt.


% % Sortieren
\phantomsection\menusubsection{2}{Sortieren}
\label{Sortieren}
Zusätzlich steht eine Sortierfunktion zum Ordnen der Suchergebnisse zur Verfügung. Wählen Sie dazu einfach die gewünschte Sortiermethode aus dem Drop-down-Menü rechts neben dem Suchfeld aus.

Folgende Sortierungen sind möglich: nach Artikelbezeichnung, nach Preis (auf- und absteigend) sowie nach Artikelnummer.


% % Kategorien
\phantomsection\menusubsection{3}{Kategorien}
\label{Kategorien}
Am linken Rand sind alle angelegten Kategorien aufgeführt.
Durch den Klick auf eine Kategorie werden alle Artikel, die zu dieser Gruppe gehören, aufgelistet. Dabei wird die Artikelsuche zurückgesetzt, die Suche wird nicht innerhalb der gewählten Kategorie fortgesetzt. Ebenso wird bei der Eingabe eines Suchbegriffes die Anzeige der Artikel einer Kategorie zurückgesetzt. Es wird folglich nicht nur die gewählte Rubrik durchsucht, sondern wiederum alle vorhandenen Kategorien.

Wird eine Kategorie gewählt, die keine Artikel enthält, wird eine entsprechende Meldung ausgegeben.

Auch die Auflistung von Artikeln einer Kategorie kann über das Drop-down-Menü sortiert werden.


% % in den Warenkorb
\phantomsection\menusubsection{4}{Zum Warenkorb hinzufügen}
\label{in den Warenkorb}
Die angezeigten Artikel können über die Schaltfläche mit dem Einkaufswagen-Symbol in den \hyperref[Warenkorb]{Warenkorb} gelegt werden. Die gewünschte Menge wird dazu im Nummernfeld des betreffenden Artikels ausgewählt.

Falls ein Artikel nicht mehr verfügbar ist, wird die Schaltfläche deaktiviert und das Nummernfeld ausgeblendet.

Weiterhin können Artikel nicht in den Warenkorb gelegt werden, wenn gleichzeitig ein \hyperref[Reklamation]{Reklamationsvorgang} läuft. Dann wird eine entsprechende Meldung angezeigt.
\newpage


% % % Reklamation
\phantomsection\menusection{3}{Reklamation}
\label{Reklamation}
Eine weitere wichtige Shopfunktion ist die Reklamation. Der Kunde besitzt die Möglichkeit, innerhalb von 14 Tagen nach Vorlage des erworbenen Produkts in unbenutztem Zustand sowie der zugehörigen Rechnung die bezahlte Geldsumme zurück\-zuerhalten.


% % Rechnung suchen
\phantomsection\menusubsection{1}{Rechnung suchen}
\label{Rechnung suchen}
Um einen Reklamationsvorgang einzuleiten, muss die Nummer der vom Kunden vorgelegten Rechnung eingegeben werden.\\
Dabei kann es zu folgenden Fehlern kommen:
\begin{itemize}
\itemsep 0em
\item Es existiert keine solche Rechnung.
\item Die gefundene Rechnung ist ihrerseits eine Reklamation.
\item Die Rechnung ist älter als 14 Tage.
\item Die Rechnungsnummer stammt von einem abgebrochenen Kaufvorgang.
\end{itemize}
Wenn zur gleichen Zeit ein Kaufvorgang stattfindet, muss dieser erst \hyperref[Kasse]{beendet} oder abgebrochen, d.\,h. der \hyperref[Artikel ändern und löschen]{Warenkorb geleert} werden, damit eine Reklamation durchgeführt werden kann. Dies wird durch eine Meldung signalisiert.


% % Rechnungsposten
\phantomsection\menusubsection{2}{Rechnungsposten in den Warenkorb legen}
\label{Rechnungsposten}
Wird die gewünschte Rechnung anhand der Rechnungsnummer gefunden, werden deren Rechnungsposten aufgelistet. Die einzelnen Positionen können wie bei der \mbox{\hyperref[in den Warenkorb]{Artikelsuche}} in den Warenkorb gelegt werden. Die Menge, die reklamiert werden kann, ist dabei selbstverständlich auf die Menge des Rechnungspostens abzüglich bisheriger Reklamationen dieser Position beschränkt. Wurde ein Rechnungsposten bereits vollständig reklamiert, wird analog zur Artikelsuche die entsprechende Schaltfläche deaktiviert und das Nummernfeld ausgeblendet.


% % Alles reklamieren
\phantomsection\menusubsection{3}{Komplette Rechnung reklamieren}
\label{Alles reklamieren}
Sollen alle Artikel einer Rechnung reklamiert werden, bietet sich der Button \textit{Alle Artikel reklamieren} an. Ein Klick darauf bewirkt, dass alle Rechnungsposten zum Warenkorb hinzugefügt werden.


% % Reklamation abbrechen
\phantomsection\menusubsection{4}{Reklamationsvorgang abbrechen}
\label{Reklamation abbrechen}
Um einen Reklamationsvorgang abzubrechen, genügt ein Klick auf die Schaltfläche \textit{Abbrechen\textit}. Dabei wird der Warenkorb geleert, die Artikelsuche freigeschaltet und es erscheint wieder das Eingabefeld für die Rechnungsnummer.
\newpage


% % % Warenkorb
\phantomsection\menusection{4}{Warenkorb}
\label{Warenkorb}
Der Warenkorb enthält die Artikel, die im \hyperref[in den Warenkorb]{Kaufvorgang} verkauft bzw. im \hyperref[Rechnungsposten]{Reklamationsvorgang}  reklamiert werden sollen. Er ist zugleich Ausgangspunkt für den Abschluss beider Vorgänge.


% % Artikel ändern und löschen
\phantomsection\menusubsection{1}{Artikel ändern und löschen}
\label{Artikel ändern und löschen}
Die im Warenkorb angezeigten Artikel können geändert und gelöscht werden.

Um die Menge eines Artikels im Warenkorb zu aktualisieren, gibt man den gewünschten Betrag in das Nummernfeld der Spalte \textit{Anzahl} ein. Nach einem Klick auf die Schaltfläche mit den zwei Pfeilen wird die Anzahl und der Gesamtpreis des Postens angepasst.

Um einen Artikel aus dem Warenkorb zu entfernen, reicht ein Klick auf den Button \textit{Löschen} in der betreffenden Zeile. Sollen alle Artikel aus dem Warenkorb gelöscht werden, klicken Sie auf \textit{Warenkorb leeren}. Falls es sich um einen Kaufvorgang handelt, wird dieser somit abgebrochen, d.\,h. es kann nun eine Reklamation gestartet werden. Dies gilt jedoch nicht für den Reklamationsvorgang. Auf der Seite \textit{Reklamation} sind weiterhin die Posten der aktuellen Rechnung aufgeführt, welche wiederum zum Warenkorb hinzugefügt werden können.


% % Kasse
\phantomsection\menusubsection{2}{Kasse und Übersicht}
\label{Kasse}
Zur (virtuellen) Kasse gelangt man, indem die Schaltfläche \textit{Zur Kasse} gewählt wird. Die Struktur der Seite unterscheidet sich beim Kaufvorgang geringfügig von der Ansicht beim Reklamationsvorgang.

Um von der Kasse zurück zum Warenkorb zu gelangen, kann bei beiden Vorgängen der Button \textit{Zurück} genutzt werden.

Ein weiteres gemeinsames Element ist die Übersicht über die gekauften bzw. zurückgegebenen Artikel. Dabei wird die Gesamtsumme am Fuß der Tabelle angezeigt.

Beim Kaufvorgang muss die Zahlungsart, mit welcher der Kunde bezahlen möchte, ausgewählt werden.  Zur Verfügung stehen Barzahlung, Kreditkarte und Lastschriftverfahren. Bei der Reklamation entfällt dieser Schritt, da die Summe immer in bar ausgezahlt wird.

Abgeschlossen und damit rechtsgültig wird der Kaufvorgang durch Betätigung der Schaltfläche \textit{Kaufen}. Abschließend erscheint nochmals eine Übersicht über die erworbenen Artikel mit Rechnungsnummer, Datum und Zahlungsart.

Eine Reklamation wird vorläufig abgeschlossen, indem die Schaltfläche \textit{Reklamationsanfrage abschicken} angeklickt wird. Damit wird eine Reklamationsanfrage an den Ladenbesitzer gesendet, der über die Ablehnung oder Genehmigung der Reklamation je nach Zustand der Ware entscheidet. Nach Absenden der Anfrage wird auch hier eine entsprechende Übersicht angezeigt.
\newpage


% % % Verwaltung
\phantomsection\menusection{5}{Verwaltung}
\label{Vewaltung}
Die Shop-Verwaltungsfunktionen, die nur dem Ladenbesitzer zur Verfügung stehen, sind unter der Rubrik \textit{Verwaltung} zusammengefasst.


% % Bisherige Verkäufe
\phantomsection\menusubsection{1}{Bisherige Verkäufe}
\label{Bisherige Verkäufe}
Eine Übersicht über alle bisherigen abgeschlossenen Verkäufe und Reklamationen ist über den Unterpunkt \textit{Verkäufe} zu erreichen. Dabei gibt es zwei verschiedene Auflistungsmöglichkeiten. In beiden Übersichten sind auch Artikel zu finden, die aus dem Sortiment gestrichen wurden.

% nach Datum geordnet
\phantomsection\menusubsubsection{1}{nach Datum geordnet}
\label{nach Datum geordnet}
In der Standardansicht werden die Rechnungen der Verkäufe und Reklamationen nach Datum geordnet aufgelistet. Neben den gekauften Artikeln werden Rechnungsnummer, Datum und Uhrzeit, Zahlungsart und Verkäufer der jeweiligen Rechnung angezeigt.

% nach Artikeln geordnet
\phantomsection\menusubsubsection{2}{nach Artikeln geordnet}
\label{nach Artikeln geordnet}
Mit einem Klick auf den Button \textit{Nach Artikeln sortieren} gelangen Sie zur zweiten Ansicht. Nun erhalten Sie einen nach Artikeln geordneten Überblick über die Verkäufe und Reklamationen. Zu jedem Artikel sind alle Rechnungen aufgelistet, in denen dieser Artikel enthalten ist. Auch hier werden Datum und Uhrzeit, Rechnungsnummer, Zahlungsart und Verkäufer aufgelistet.

Nicht mehr im Sortiment enthaltene Artikel erkennen Sie an der Information \textit{nicht mehr im Sortiment} neben der Artikelnummer.

% XML-Export
\phantomsection\menusubsubsection{3}{XML-Export}
\label{XML-Export}
Die Schaltfläche \textit{Als XML exportieren} veranlasst, dass die Verkaufsdaten in eine XML-Datei mit der Struktur der momentanen Auflistungsart geschrieben werden.
\newpage


% % Sortimentverwaltung
\phantomsection\menusubsection{2}{Sortimentverwaltung}
\label{Sortimentverwaltung}
Der Unterpunkt \textit{Sortiment} in der Rubrik \textit{Verwaltung} erlaubt die Verwaltung des Sortiments und des Lagers in einem. Hier sind alle angebotenen Artikel und ihre zugehörigen Unter- und Oberkategorien aufgeführt. Zu den Artikeln werden jeweils Artikelnummer, aktueller Lagerbestand und Stückpreis angezeigt.

Es können sämtliche Artikel und Kategorien bearbeitet und gelöscht werden sowie Artikel und Kategorien neu ins Sortiment aufgenommen werden.

% % % Kategorie bearbeiten
\phantomsection\menusubsubsection{1}{Kategorie bearbeiten}
\label{Kategorie bearbeiten}
Um eine Kategorie zu bearbeiten, genügt ein Klick auf den Button \textit{Bearbeiten} in der betreffenden Zeile. Nun kann ein neuer Name eingegeben werden. Dabei ist darauf zu achten, dass kein Name gewählt wird, der bereits einer anderen Kategorie zugewiesen wurde.

Falls es sich bei der gewählten Kategorie um eine Unterkategorie handelt, besteht zudem die Möglichkeit, die dazugehörige Oberkategorie zu ändern.

Durch einen Klick auf \textit{Speichern} werden die Änderungen übernommen, sofern ein gültiger Name eingegeben und ggf. eine Oberkategorie ausgewählt wurde.

% Artikel bearbeiten
\phantomsection\menusubsubsection{2}{Artikel bearbeiten}
\label{Artikel bearbeiten}
Zum Formular für die Änderung der Artikeldaten gelangt man durch Anwählen der entsprechenden Schaltfläche \textit{Bearbeiten}. Es erscheinen Eingabefelder für Artikelbezeichnung, Kategorie, Stückpreis in Euro, minimale Stückzahl im Lager und tatsächlicher Lagerbestand. Die Artikelnummer kann nachträglich nicht mehr geändert werden.\\[.3cm]
Beim Ändern der Daten ist Folgendes zu beachten:
\begin{itemize}
\itemsep 0em
\item Der Artikelname darf nicht leer sein.
\item Der Stückpreis muss in der üblichen Notation angegeben werden,\\
z.\,B. \texttt{1.234,56~€}, \texttt{123~€}, \texttt{0,90~€}. Das Eurozeichen kann entfallen.
\item Minimale Stückzahl und tatsächlicher Lagerbestand müssen positive Zahlen sein.
\item Es muss eine Kategorie ausgewählt sein.
\end{itemize}
Der Button \textit{Speichern} führt nach bestandener Gültigkeitsprüfung zur Speicherung der Änderungen.

% Artikel/Kategorie löschen
\phantomsection\menusubsubsection{3}{Artikel/Kategorie löschen}
\label{Artikel/Kategorie löschen}
Um eine Kategorie oder ein Artikel aus dem Sortiment zu entfernen, reicht der Klick auf die Schaltfläche \textit{Löschen} in der entsprechenden Zeile. Bitte beachten Sie, dass falls es sich um eine Kategorie handelt, alle in ihr befindlichen Kategorien und Artikel mit gelöscht werden! Die gelöschten Artikel erscheinen jedoch weiterhin in der \hyperref[Bisherige Verkäufe]{Liste der bisherigen Verkäufe}. Zudem ist es nach wie vor möglich, gelöschte Artikel zu reklamieren.
\newpage

% Kategorie hinzufügen
\phantomsection\menusubsubsection{4}{Kategorie hinzufügen}
\label{Kategorie hinzufügen}
Um eine neue (Unter-)Kategorie zu erstellen, wählen Sie \textit{Kategorie hinzufügen} bzw. \textit{Unterkategorie hinzufügen}. Geben Sie einen passenden, nichtleeren Namen ein und wählen Sie ggf. eine Oberkategorie aus. Zum Abschluss des Vorgangs betätigen Sie \textit{Speichern}.

% Artikel hinzufügen
\phantomsection\menusubsubsection{5}{Artikel hinzufügen}
\label{Artikel hinzufügen}
Soll ein neuer Artikel erstellt werden, gehen Sie auf \textit{Artikel hinzufügen}. Das Formular ähnelt dem beim \hyperref[Artikel bearbeiten]{Bearbeiten von Artikeln}. Der einzige Unterschied besteht darin, dass auch die Artikelnummer angegeben werden muss.\\[.3cm]
Bitte beachten Sie auch hier folgende Hinweise:
\begin{itemize}
\itemsep 0em
\item Der Artikelname darf nicht leer sein.
\item Die Artikelnummer muss eine positive Zahl sein, die noch keinem anderen Artikel zugewiesen wurde.
\item Der Stückpreis muss in der üblichen Notation angegeben werden,\\
z.\,B. \texttt{1.234,56~€}, \texttt{123~€}, \texttt{0,90~€}. Das Eurozeichen kann entfallen.
\item Minimale Stückzahl und tatsächlicher Lagerbestand müssen positive Zahlen sein.
\item Es muss eine Kategorie ausgewählt sein.
\end{itemize}
Nach einem Klick auf die Schaltfläche \textit{Speichern} werden die Eingaben auf Gültigkeit überprüft. Im Erfolgsfall wird der Artikel dem Sortiment hinzugefügt.
\newpage


% % Personalverwaltung
\phantomsection\menusubsection{3}{Personalverwaltung}
\label{Personalverwaltung}
Unter dem Punkt \textit{Personal} in der Rubrik \textit{Verwaltung} findet man eine Auflistung der Angestellten und personalspezifische Verwaltungsfunktionen, darunter die Funktion zum Ändern der Passwortregeln.

% Profil ansehen und bearbeiten
\phantomsection\menusubsubsection{1}{Profil ansehen und bearbeiten}
\label{Profil ansehen und bearbeiten}
Für das Einsehen und Ändern des Profils eines Angestellten gibt es die Schaltfläche \textit{Ansehen/Bearbeiten}. Die angezeigte Übersicht sowie die Formulare zur Änderung der persönlichen Daten und des Passwortes des betreffenden Angestellten entsprechen denen des \hyperref[Mein Profil]{eigenen Profils}.

% Mitarbeiter entlassen
\phantomsection\menusubsubsection{2}{Mitarbeiter entlassen}
\label{Mitarbeiter entlassen}
Neben dem Button \textit{Ansehen/Bearbeiten} gibt es eine Schaltfläche \textit{Entlassen}, die zur Auflösung des betreffenden Angestelltenprofils führt. Diesem Mitarbeiter ist es infolgedessen nicht mehr möglich, sich einzuloggen. Dessen Name steht jedoch weiterhin bei den von diesem Angestellten getätigten Verkäufen.

Um alle Angestellten auf einmal zu entlassen, steht der Button \textit{Alle Angestellten entlassen} unterhalb der Auflistung zur Verfügung.

% Mitarbeiter einstellen
\phantomsection\menusubsubsection{3}{Mitarbeiter einstellen}
\label{Mitarbeiter einstellen}
Um ein Profil für einen neuen Angestellten zu erstellen, existiert die Funktion \textit{Einstellen}. Das Formular ist das gleiche wie bei der Änderung der \hyperref[Persönliche Daten ändern]{eigenen persönlichen Daten}.
Beim Feld \textit{Benutzername}, das sonst deaktiviert ist, muss darauf geachtet werden, dass der Benutzername des neuen Mitarbeiters länger als zwei Zeichen ist und nicht bereits ein anderer Nutzer diesen Benutzername besitzt.

Nach Anlegen des neuen Nutzerprofils mittels \textit{Speichern} bekommt der Ladenbesitzer eine \hyperref[Allgemeine Benachrichtigungen]{Benachrichtigung, die das Erstpasswort des neuen Mitarbeiters enthält}.
Mit diesem kann sich der Angestellte anmelden, wird jedoch sofort aufgefordert, \hyperref[Authentifikation]{sein Passwort zu ändern}.

% Passwortregeln ändern
\phantomsection\menusubsubsection{4}{Passwortregeln ändern}
\label{Passwortregeln ändern}
Die Passwortregeln gewährleisten die Sicherheit der Passwörter, indem jedes Passwort vor einer Änderung entsprechend der durch den Ladenbesitzer festgelegten Sicherheitsregeln auf Gültigkeit geprüft wird. Mit der Funktion \textit{Passwortregeln ändern} können diese Regeln geändert werden. Der Ladenbesitzer kann festlegen, ob Groß- und Kleinbuchstaben, Zahlen und Sonderzeichen in einem Passwort vorkommen müssen. Zudem kann er die minimale Länge festsetzen, die jedes Passwort mindestens vorweisen muss. Eine Mindestlänge von wenigstens sechs Zeichen ist jedoch fest vorgegeben.

Unabhängig von den Einstellungen des Ladenbesitzers müssen Passwörter mindestens einen Buchstaben enthalten; des Weiteren dürfen keine Leerraumzeichen in einem Passwort vorkommen.
\newpage


% % Witzeverwaltung
\phantomsection\menusubsection{4}{Witzeverwaltung}
\label{Witzeverwaltung}
Der letzte Punkt in den Verwaltungsfunktionen ist die \textit{Witzeverwaltung}. Darin werden die Witze verwaltet, die den Nutzern nach dem Login als Begrüßung angezeigt werden.

% Witze ändern
\phantomsection\menusubsubsection{1}{Witze ändern}
\label{Witze ändern}
Soll ein Witz geändert werden, gelangt man durch einen Klick auf den entsprechenden Button \textit{Ändern} zum Änderungsformular. Im Textfeld kann der Text des Witzes editiert werden. Wählen Sie \textit{Speichern}, um die Änderungen zu übernehmen. Falls kein Text eingegeben wurde, wird eine Fehlermeldung angezeigt.

% Witze löschen
\phantomsection\menusubsubsection{2}{Witze löschen}
\label{Witze löschen}
Einen Witz zu entfernen, erreicht man durch Anwählen der Schaltfläche \textit{Löschen} in der jeweiligen Zeile.
Sollen alle Witze gelöscht werden, wählen Sie \textit{Alle Witze löschen}. In diesem Fall wird dem Nutzer nach dem Login statt des Begrüßungswitzes eine entsprechende Nachricht angezeigt.

% Witz hinzufügen
\phantomsection\menusubsubsection{3}{Witz hinzufügen}
\label{Witz hinzufügen}
Wählen Sie \textit{Witz hinzufügen}, um einen neuen Witz anzulegen. Nachdem Sie einen Text eingegeben haben, gehen Sie auf \textit{Witz speichern}. Ab sofort fließt der neue Witz in die Ermittlung des Begrüßungswitzes ein.
\newpage


% % % Benachrichtigungen
\phantomsection\menusection{6}{Benachrichtigungen}
\label{Benachrichtigungen}
Unter dem Punkt \textit{Benachrichtigungen} sind alle Mitteilungen an den Ladenbesitzer zusammengefasst.

Nachfolgend werden die verschiedenen Arten von Nachrichten beleuchtet.


% % Allgemeine Benachrichtigungen
\phantomsection\menusubsection{1}{Allgemeine Benachrichtigungen}
\label{Allgemeine Benachrichtigungen}
Allgemeine Benachrichtigungen bekommt der Ladenbesitzer, wenn:
\begin{itemize}
\itemsep 0pt
\item ein Angestellter seine persönlichen Daten oder sein Passwort geändert hat.
\item ein Angestellter hinzugefügt wurde. Die Nachricht enthält in diesem Fall das Startpasswort des neuen Mitarbeiters.
\item der Lagerbestand eines Artikels die untere Grenze unterschritten hat.
\end{itemize}
Mitteilungen dieser Art können mittels Klick auf \textit{Löschen} entfernt werden.


% % Aufforderung zur Passwortänderung
\phantomsection\menusubsection{2}{Aufforderung zur Passwortänderung}
\label{Aufforderung zur Passwortänderung}
Von dieser Mitteilungsart gibt es immer nur eine Benachrichtigung. Diese erscheint, wenn das Passwort des Ladenbesitzers nicht den Sicherheitsregeln entspricht. Sie kann nicht gelöscht werden, sondern verschwindet erst, wenn der Ladenbesitzer \hyperref[Passwort ändern]{sein Passwort den Passwortregeln angepasst} hat.


% % Genehmigung von Reklamationen
\phantomsection\menusubsection{3}{Genehmigung von Reklamationen}
\label{Genehmigung von Reklamationen}
Hat ein Kunde Artikel zurückgegeben, muss der Ladenbesitzer die Reklamation zunächst genehmigen. Dazu bekommt er eine Mitteilung, welche die betroffenen Artikel nach Betätigung der Schaltfläche \textit{Ansehen} auflistet. Stimmt der Ladenbesitzer der Reklamation zu, wählt er die Schaltfläche \textit{Genehmigen}, ansonsten ist \textit{Abweisen} zu wählen.

Nachrichten dieser Art können nicht gelöscht werden und werden erst nach einer Genehmigung oder Abweisung entfernt.
\newpage


% % % Mein Profil
\phantomsection\menusection{7}{Mein Profil}
\label{Mein Profil}
Der Abschnitt \textit{Mein Profil} beinhaltet die Verwaltung der persönlichen Daten des angemeldeten Nutzers und steht allen Akteuren des Shops zur Verfügung. 

Die Startseite des Profils enthält eine Übersicht über die persönlichen Daten des Nutzers.


% % Persönliche Daten ändern
\phantomsection\menusubsection{1}{Persönliche Daten ändern}
\label{Persönliche Daten ändern}
Über den Button \textit{Daten bearbeiten} erreicht man das Formular zum Ändern der Daten. In der folgenden Tabelle sind die Eingabefelder mit Bemerkungen zu den zugelassenen Eingaben aufgeführt.

\vspace{0.4cm}
\begin{tabularx}{\textwidth}{lX}
\toprule
\textbf{Eingabefeld} & \textbf{Bemerkung zu den zulässigen Eingaben} \\
\midrule
Vorname & Dieses Feld darf nicht leer sein. \\[.3cm]
Nachname & Dieses Feld darf nicht leer sein. \\[.3cm]
Benutzername & Dieses Feld ist deaktiviert, da der Benutzername nachträglich nicht mehr verändert werden kann. \\[.3cm]
E-Mail-Adresse & Dieses Feld darf nicht leer sein und muss eine gültige E-Mail-Adresse enthalten. \\[.3cm]
Geschlecht & Es muss ein Eintrag aus der Liste ausgewählt sein. \\[.3cm]
Geburtsdatum & Dieses Feld muss ein Datum, das in der Vergangenheit liegt, enthalten. Das Datum muss in der im deutschsprachigen Raum üblichen Notation eingegeben werden, z.\,B. \texttt{14.03.85} oder \texttt{29.02.1912}. \newline Falls nur zwei Ziffern für das Jahr angegeben werden, wird das Jahr automatisch anhand das aktuellen Jahrhunderts ergänzt, allerdings so, dass das Jahr in der Vergangenheit liegt. Aus \texttt{xx.xx.99} wird beispielsweise \texttt{xx.xx.1999}. \\[.3cm]
Familienstand & Es muss ein Eintrag aus der Liste ausgewählt sein. \\[.3cm]
Telefonnummer & Es muss eine gültige Telefonnummer eingegeben werden. \\[.3cm]
Straße & Dieses Feld darf nicht leer sein. \\[.3cm]
Hausnummer & Dieses Feld darf nicht leer sein und muss mindestens eine Ziffer enthalten. \\[.3cm]
PLZ & Die Postleitzahl muss aus genau fünf Ziffern bestehen. \\[.3cm]
Ort & Dieses Feld darf nicht leer sein. \\
\bottomrule
\end{tabularx} \\[.5cm]

Nach einem Klick auf \textit{Speichern}  werden die Eingaben validiert. Falls alles ordnungsgemäß eingegeben wurde, werden die Daten übernommen. Ansonsten erscheinen an den ungültigen Eingabefeldern entsprechende Fehlermeldungen.
\newpage

% % Passwort ändern
\phantomsection\menusubsection{2}{Passwort ändern}
\label{Passwort ändern}
Über den Button \textit{Passwort ändern} auf der Profilseite ist es möglich, sein Passwort zu ändern. Dabei ist Folgendes zu beachten:
\begin{itemize}
\itemsep 0pt
\item Das alte Passwort muss korrekt eingegeben werden.
\item Das neue Passwort muss den angezeigten Sicherheitsregeln entsprechen.
\item Das neue Passwort muss erneut eingegeben werden. Beide Passwörter müssen übereinstimmen.
\end{itemize}
Um die Passwortänderung auszuführen, klicken Sie auf \textit{Passwort ändern}. Wenn keine Fehlermeldungen an den Eingabefeldern erscheinen, wurde Ihr Passwort erfolgreich geändert.

\end{document}